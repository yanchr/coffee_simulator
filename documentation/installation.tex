\chapter{Installation}\label{ch:installation}
In diesem Kapitel sind die erforderlichen Schritte um die Applikation lokal laufen zu lassen kurz beschrieben.


\section{Backend}\label{sec:backend}
Um das Backend zu starten, muss zuerst die Camunda Workflow Engine gestartet werden.
Am einfachsten geht dies über Docker mit dem folgenden Befehl:

\begin{minted}{bash}
docker run -d --name camunda -p 8080:8080 camunda/camunda-bpm-platform:7.15.0
\end{minted}

Nach einiger Zeit sollte das Camunda Control Panel im Browser über \mintinline{text}{http://localhost:8080/camunda} erreichbar sein (Credentials: demo:demo).

Nun müssen nur noch die beiden Files \mintinline{text}{diagrams/ingredients.dmn} und \mintinline{text}{diagrams/coffee_simulator.bpmn} aus dem Modeler heraus deployed und der Prozess ``Coffee\_Process'' gestartet werden.


\section{Frontend}\label{sec:frontend}
Um das Frontend auszuführen, muss nur die HTML--Datei \mintinline{text}{index.html} in einem Browser geöffnet werden.
Falls die Adresse des Backends nicht \mintinline{text}{localhost:8080} ist, kann diese in der Datei \mintinline{text}{backend.js} über die Variable \mintinline{text}{backendBaseUrl} konfiguriert werden.

\textbf{Wichtig: CORS mit Camunda}:
Die Requests des Frontends werden durch die Same-Origin-Policy \footnote{\url{https://developer.mozilla.org/de/docs/Web/HTTP/CORS}} des Browsers geblockt.
Um dieses Problem zu beheben, müssten im Backend (Camunda Workflow Engine) entsprechende Header gesetzt werden.
Siehe evtl. \footnote{\url{https://forum.camunda.org/t/enable-cors-in-camunda-tomcat-7-4/1341/4}}.
Als Workaround kann der Browser ohne Sicherheitsfeatures gestartet werden:

\begin{minted}{bash}
# Chrome auf Linux
google-chrome --disable-web-security --user-data-dir=<Beliebiger Ordner>
\end{minted}

\begin{minted}{bash}
# Chrome auf Windows
chrome.exe --user-data-dir="<Beliebiger Ordner>" --disable-web-security
\end{minted}


