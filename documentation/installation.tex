\chapter{Installation}\label{ch:installation}
In diesem Kapitel sind die erforderlichen Schritte um die Applikation lokal laufen zu lassen kurz beschrieben.


\section{Backend}\label{sec:backend}
Um das Backend zu starten, muss zuerst die Camunda Workflow Engine gestartet werden.
Am einfachsten geht dies über Docker mit dem folgenden Befehl:

\begin{minted}{bash}
docker run -d --name camunda -p 8080:8080 camunda/camunda-bpm-platform:7.15.0
\end{minted}

Nach einiger Zeit sollte das Camunda Control Panel im Browser über \mintinline{text}{http://localhost:8080/camunda} erreichbar sein (Credentials: demo:demo).

Nun müssen nur noch die beiden Files \mintinline{text}{diagrams/ingredients.dmn} und \mintinline{text}{diagrams/coffee_simulator.bpmn} aus dem Modeler heraus deployed und der Prozess ``Coffee\_Process'' gestartet werden.


\section{Frontend}\label{sec:frontend}
Um das Frontend auszuführen, muss nur die HTML--Datei \mintinline{text}{index.html} in einem Browser geöffnet werden.
Falls die Adresse des Backends nicht \mintinline{text}{localhost:8080} ist, kann diese in der Datei %TODO definiert werden.
