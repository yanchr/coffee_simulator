\chapter{Vorgehen und Probleme}\label{ch:vorgehen-und-probleme}

Wir haben uns zuerst alle Aufgaben überlegt, die wir abbilden möchten.
Als wir dann damit begannen, unsere Überlegungen als BPMN aufzuzeichnen, hatten wir anfänglich etwas Mühe damit, die Einteilung in die verschiedenen Lanes zu machen.
Zuerst wollten wir den Benutzer als eigenes System auch in das Diagramm integrieren.
Wir haben aber relativ schnell gemerkt, dass es mit unserem Setup mit dem Frontend nicht sinnvoll ist, das Verhalten des Benutzers in der Workflow Engine abzubilden.
Stattdessen haben wir uns dafür entschieden, nur die internen Tasks der Maschine abzubilden und eine REST-Schnittstelle zur Verfügung zu stellen, über die mit der Maschine kommuniziert werden kann.

Eine Schwierigkeit war insbesondere der Standby-Zustand, der sich endlos wiederholt.
Da die Workflow Engine (noch) keine Unterstützung für sich wiederholende Subtasks bietet, mussten wir den Subtask auf sich selbst weiterleiten, um so eine Endlosschleife zu erhalten.
Damit bewirkten wir aber ein scheinbar unberechenbares Verhalten und eine Überlastung unseres PCs.
Deshalb begannen wir mit ``delays'' in den Script-Tasks bzw. später mit Timern zu arbeiten, um die endlose Ausführung des Tasks besser kontrollieren zu können.
Insgesamt mussten wir viel ausprobieren und wieder anpassen, bis der Prozess sauber in der Workflow Engine ausgeführt werden konnte.